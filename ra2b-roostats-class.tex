    \documentclass[11pt]{article}
    \usepackage{graphicx}
    
    \begin{document}
    
    \title{Exact Likelihood for CMS SUSY RA2b}
    \author{O. Long, H. Prosper, S. Sekmen}
    \date{\today}
    \maketitle
    
    \section{Goal}
    
    Calculate explicitly the likelihood function for the RA2b analysis:
    \begin{itemize}
    \item Identify the observations
    \item Identify the probabilities for each datum (e.g., Poisson for a count and Gaussian for other observations)
    \item Identify how each parameter pertains to the observations
    \end{itemize}
    
    The basic design principles are: 1) partition the data into disjoint sets so that the joint likelihood is the product of the likelihoods for each set and 2) explicitly model all the relationships between the parameters of the joint likelihood. For the RA2b analysis, this requires that each 3-jet mass bin in the SL sample be partitioned into \emph{three} MET regions, namely MSB---Medium Sideband---(50$<$MET$<$100), SB (100$<$MET$<$150) and SIG (150$<$MET).
    
    \section{Data}
    The data are counts from direct observation.
    Since it is important to keep a clear distinction between data and parameters,
    we use the following convention: data are denoted by upper case symbols, e.g.
    $N$, and parameters by lower case symbols, e.g. $n$.  
    Each data count is included in the likelihood with its own Poisson PDF.
    %
    \begin{eqnarray*}
    SIG & : & N_{SIG} \\
    SB & : & N_{SB,i} \quad \textrm {where $i$ is a 3-jet mass bin} \\ 
    A & : & N_A \\
    D & : & N_D \\ 
    LSB & : & N_{LSB,i} \quad \textrm {where $i$ is a 3-jet mass bin} \\ \\
    SIG-SL & : & N_{SIG-SL,i} \quad \textrm {where $i$ is a 3-jet mass bin} \\
    SB-SL & : & N_{SB-SL,i} \quad \textrm {where $i$ is a 3-jet mass bin} \\
    MSB-SL & : & N_{MSB-SL,i} \quad \textrm {where $i$ is a 3-jet mass bin} \\
    \end{eqnarray*}
    %
    We also rely on the following inputs from the simulation (MC).
    Since the counts in the MC samples are actually weighted events,
    with a continuum of weight factors for the QCD MC, we treat the MC
    inputs differently.
    The following QCD MC inputs are included in the likelihood with
    Gaussian PDFs, where the width of the Gaussian is taken as  $\sigma = \sqrt{\sum_j w_j^2}$,
    where $w_j$ is the weight for simulated event $j$.
    %
    \begin{eqnarray*}
    SIG-MC^{QCD} & : & N_{SIG-MC}^{QCD} \\
    SB-MC^{QCD} & : & N_{SB-MC}^{QCD} \\
    A-MC^{QCD} & : & N_{A-MC}^{QCD} \\
    D-MC^{QCD} & : & N_{D-MC}^{QCD} \\ 
    \end{eqnarray*}
    %



    \noindent
    All of the following inputs described below are fixed in the likelihood. \\


    \noindent
    The electroweak (EW) backgrounds are taken as fixed parameters that are
    known.  Uncertainties on the EW background inputs will be dealt with
    in the evaluation of systematic uncertainties.
    %
    \begin{eqnarray*}
    SIG-MC^{EW} & : & N_{SIG-MC}^{EW} \\ 
    SB-MC^{EW} & : & N_{SB-MC,i}^{EW}  \quad \textrm {where $i$ is a 3-jet mass bin} \\ 
    A-MC^{EW} & : & N_{A-MC}^{EW} \\
    D-MC^{EW} & : & N_{D-MC}^{EW} \\ 
    SIG-SL-MC^{EW} & : & N_{SIG-SL-MC,i}^{EW}  \quad \textrm {where $i$ is a 3-jet mass bin} \\ 
    SB-SL-MC^{EW} & : & N_{SB-SL-MC,i}^{EW}  \quad \textrm {where $i$ is a 3-jet mass bin} \\ 
    MSB-SL-MC^{EW} & : & N_{MSB-SL-MC,i}^{EW}  \quad \textrm {where $i$ is a 3-jet mass bin} \\ 
    \end{eqnarray*}
    %
    In the ABCD method for the QCD background, the small amount of $t\bar t$ in the
    A and D regions is included.
    The $t\bar t$ counts in A and D are fixed to the MC predictions..
    %
    \begin{eqnarray*}
    A-MC^{t\bar t} & : & N_{A-MC}^{t\bar t} \\
    D-MC^{t\bar t} & : & N_{D-MC}^{t\bar t} \\
    \end{eqnarray*}
    %
    Finally, signal contamination of the data control samples is taken from
    the signal MC.
    When testing a particular SUSY hypothesis (say a point in the usual 2-D CMSSM plane),
    we include the predicted contributions to the data control samples,
    for that particular set of SUSY model parameters,
    as fixed parameters.
    When testing the no-signal hypothesis, all signal contributions to the data control
    samples (everything but the SIG region) are fixed to zero.
    %
    \begin{eqnarray*}
    %SIG-MC^{s} & : & N_{SIG-MC}^{s} \\ 
    %SB-MC^{s} & : & N_{SB-MC,i}^{s} \\ \\
    SB-MC^{s} & : & N_{SB-MC,i}^{s} \quad \textrm {where $i$ is a 3-jet mass bin} \\ 
    A-MC^{s} & : & N_{A-MC}^{s} \\
    D-MC^{s} & : & N_{D-MC}^{s} \\ 
    LSB-MC^{s} & : & N_{LSB-MC,i}^{s} \quad \textrm {where $i$ is a 3-jet mass bin} \\ \\
    SIG-SL-MC^{s} & : & N_{SIG-SL-MC,i}^{s} \quad \textrm {where $i$ is a 3-jet mass bin} \\
    SB-SL-MC^{s} & : & N_{SB-SL-MC,i}^{s} \quad \textrm {where $i$ is a 3-jet mass bin} \\
    MSB-SL-MC^{s} & : & N_{MSB-SL-MC,i}^{s} \quad \textrm {where $i$ is a 3-jet mass bin} \\
    \end{eqnarray*}
    %
    
    
    
    
    
    
    
    
    
  %%==============================================================================================
    
    
    \section{Building the likelihood}
    
    \subsection{Step 1: Build the model for each region (i.e., box)}
    
    The $n$s are the expected (i.e., true) counts per box or bin.
    
    \begin{eqnarray}
    n_{SIG} & = & \mu_{SIG}^{t\bar{t}} + \mu_{SIG}^{QCD} + \mu_{SIG}^{EW} + s_{SIG} \\
    n_{SB,i} & = & \mu_{SB,i}^{t\bar{t}} + \mu_{SB,i}^{QCD} + \mu_{SB,i}^{EW} + s_{SB,i} \\
    n_{A} & = & \mu_{A}^{QCD} + \mu_{A}^{t\bar t} + \mu_{A}^{EW} + s_{A}\\
    n_{D} & = & \mu_{D}^{QCD} + \mu_{D}^{t\bar t} + \mu_{D}^{EW} + s_{D}\\
    n_{LSB,i} & = & \mu_{LSB,i}^{QCD} \\ 
    \nonumber \\
    n_{SIG-SL,i} & = & \mu_{SIG-SL,i}^{t\bar{t}} + \mu_{SIG-SL,i}^{EW} + s_{SIG-SL,i} \\  
    n_{SB-SL,i} & = & \mu_{SB-SL,i}^{t\bar{t}} + \mu_{SB-SL,i}^{EW} + s_{SB-SL,i} \\
    n_{MSB-SL,i} & = & \mu_{MSB-SL,i}^{t\bar{t}} + \mu_{MSB-SL,i}^{EW} + s_{MSB-SL,i}  \\
    \nonumber \\
    n_{SIG-MC}^{QCD} & = & \mu_{SIG-MC}^{QCD} \\
    n_{SB-MC}^{QCD} & = & \mu_{SB-MC}^{QCD} \\
    n_{A-MC}^{QCD} & = & \mu_{A-MC}^{QCD} \\
    n_{D-MC}^{QCD} & = & \mu_{D-MC}^{QCD}
    \end{eqnarray}
    \noindent
    We then determine how the parameters are related to the data for each box or bin.  For example $N_{SIG}$ and $\mu_{SIG}^{t\bar{t}} + \mu_{SIG}^{QCD} + \mu_{SIG}^{EW} + s$ are related by a Poisson in which the parameter is the Poisson mean.
    




  %%==============================================================================================
    \subsection{Step 2: Determine the relationships between all parameters}
    
    We'll write everything in terms of the quantities in the SIG region.
    Here is the relationship between the $t\bar{t}$ parameters in the SB and SIG regions:
    \begin{equation}
    \mu_{SB}^{t\bar{t}} = \mu_{SIG}^{t\bar{t}} \left( \frac{\mu_{SB-SL}^{t\bar{t}}}{\mu_{SIG-SL}^{t\bar{t}}} \right),
    \end{equation}
    where
    \begin{eqnarray}
    \mu_{SIG-SL}^{t\bar{t}} = \sum_i \mu_{SIG-SL,i}^{t\bar{t}}, \\
    \mu_{SB-SL}^{t\bar{t}} = \sum_i \mu_{SB-SL,i}^{t\bar{t}} 
    \end{eqnarray}
    and $i$ is a 3-jet mass bin.
    
    Here is the relationship between the QCD parameters in the SIG and other regions:
    \begin{equation}
    \mu_{SIG}^{QCD} = \mu_{SIG-MC}^{QCD} \left(\frac{\mu_{A-MC}^{QCD}}{\mu_{A}^{QCD}} \right) \left(\frac{\mu_{SB}^{QCD}}{\mu_{SB-MC}^{QCD}} \right) \left(\frac{\mu_{D}^{QCD}}{\mu_{D-MC}^{QCD}} \right),
    \end{equation}
    which we re-write in terms of $\mu_{SIG}^{QCD}$:
    \begin{equation}
    \mu_{SB}^{QCD} = \mu_{SB-MC}^{QCD} \left(\frac{\mu_{SIG}^{QCD}}{\mu_{SIG-MC}^{QCD}} \right) \left(\frac{\mu_{A}^{QCD}}{\mu_{A-MC}^{QCD}} \right) \left(\frac{\mu_{D-MC}^{QCD}}{\mu_{D}^{QCD}} \right).
    \end{equation}
    The above expressions follow directly from the definition of the bias correction.
    
    Defining the spectral fractions 
    \begin{equation}
    f_{k,i}^{j} \equiv \frac{\mu_{k,i}^{j}}{\sum_{i} \mu_{k,i}^{j}},
    \end{equation}
    where $\mu_{k,i}^{j}$ is the expected count in the $i$th bin of the 3-jet mass spectrum of region $k$ and process $j$,
    we can write
    \begin{eqnarray}
    \mu_{SB,i}^{QCD} & = & f_{LSB,i}^{QCD} \ \ \mu_{SB}^{QCD}, \\
    \mu_{SB,i}^{t\bar{t}} & = & f_{(MSB+SB+SIG),SL,i}^{t\bar{t}} \ \ \mu_{SB}^{t\bar{t}}. 
    \end{eqnarray}
    The symbol $MSB+SB+SIG$ implies a sum over the three regions.
    The EW and signal are handled in an analogous way, except that the spectral fractions are constrained by MC calculations.
    \begin{eqnarray}
    \mu_{SB,i}^{EW} & = & f_{SB-MC,i}^{EW} \ \mu_{SB-MC}^{EW} \\
    s_{SB,i} & = & f_{SB-MC,i}^{s} \ \ s_{SB-MC}. 
    \end{eqnarray}
    
    \subsection{Step 3: Build the likelihood}
    
    The basic rule is that data must appear only once in the joint likelihood.
    
    \begin{eqnarray*}
    L_{SIG} & = & P(\ N_{SIG} \ | \ n_{SIG}\ ), \\
    L_{SB} & = & \prod_i P(\ N_{SB,i} \ | \ n_{SB,i}\ ), \\
    L_{SIG-SL\&SB-SL\&MSB-SL} & = & \prod_i P(\ N_{SIG-SL,i} \ | \ n_{SIG-SL,i}\  ), \\
               & \times & \prod_i P(\ N_{SB-SL,i} \ | \ n_{SB-SL,i}\  ) \\
               & \times & \prod_i P(\ N_{MSB-SL,i} \ | \ n_{MSB-SL,i}\  ), \\
    L_{LSB} & = & \prod_i P(\ N_{LSB,i} \ | \ n_{LSB,i} ), \\
    L_{A\&D\&SB\&SIG} & = & G(\ N_{A-MC} \ | \ n_{A-MC}^{QCD} \, ,\, \sigma_{A-MC}^{QCD} \ ) \ P(N_{A} \ | \ n_{A}\ ) \\ 
                & \times &  G(\ N_{D-MC} \ | \ n_{D-MC}^{QCD} \, ,\, \sigma_{D-MC}^{QCD} \ ) \ P(N_{D} \ | \ n_{D}\ ) \\
                & \times &  G(\ N_{SB-MC} \ | \ n_{SB-MC}^{QCD} \, , \, \sigma_{SB-MC}^{QCD} \ ) \\
                & \times &  G(\ N_{SIG-MC} \ | \ n_{SIG-MC}^{QCD}\, , \, \sigma_{SIG-MC}^{QCD} \ ) \\
    \end{eqnarray*}
    
%%  \begin{eqnarray*}
%%  L_{SIG} & = & P(N_{SIG} | \mu_{SIG}^{t\bar{t}} + \mu_{SIG}^{QCD} + \mu_{SIG}^{EW} + s_{SIG}), \\
%%  L_{SB} & = & \prod_i P(N_{SB,i} | \mu_{SB,i}^{t\bar{t}} + \mu_{SB,i}^{QCD} + \mu_{SB,i}^{EW} + s_{SB,i}), \\
%%  L_{SIG-SL\&SB-SL\&MSB-SL} & = & \prod_i P(N_{SIG-SL,i} | \mu_{SIG-SL,i}^{t\bar{t}} ), \\
%%             & \times & \prod_i P(N_{SB-SL,i} | \mu_{SB-SL,i}^{t\bar{t}} ) \\
%%             & \times & \prod_i P(N_{MSB-SL,i} | \mu_{MSB-SL,i}^{t\bar{t}} ), \\
%%  L_{LSB} & = & \prod_i P(N_{LSB,i} | \mu_{LSB,i}^{QCD} ), \\
%%  L_{A\&D\&SB\&SIG} & = & P(N_{A-MC} | \mu_{A-MC}^{QCD}) P(N_{A} | \mu_{A}^{QCD}) \\ 
%%                    & \times &  P(N_{D-MC} | \mu_{D-MC}^{QCD}) P(N_{D} | \mu_{D}^{QCD}) \\
%%                    & \times &  P(N_{SB-MC} | \mu_{SB-MC}^{QCD}) \\
%%                    & \times &  P(N_{SIG-MC} | \mu_{SIG-MC}^{QCD}) \\
%%  L_{SB-MC}^{EW} & = & P(N_{SB-MC,i}^{EW} | \mu_{SB-MC,i}^{EW}) \\
%%  L_{SB-MC}^{s} & = & P(N_{SB-MC,i}^{s} | s_{SB-MC,i}) \\
%%  L_{SIG-MC}^{EW} & = & P(N_{SIG-MC}^{EW} | \mu_{SIG-MC}^{EW}) \\
%%  L_{SIG-MC}^{s} & = & P(N_{SIG-MC}^{s} | s_{SIG-MC}) 
%%  \end{eqnarray*}
    

    \section{ Implementation }


     A first attempt at encoding the likelihood exists in this package.
     The relevant source code is {\tt ra2bRoostatsClass.c}.


    \end{document}




